\chapter{Conclusioni}
\label{chap:conclusioni}
Lo scopo di questa tesi era quella di sviluppare un editor di oggetti 3D configurabile, 
e che usasse la realtà aumentata.Con gli argomenti trattati abbiamo visto come negli 
ultimi anni la tendenza sia quella di usare il linguaggio JavaScript non per l'esecuzione 
di pochi controlli nelle pagine web come veniva fatto nei primi anni successivi alla 
sua nascita, ma come un linguaggio di riferimento per lo sviluppo di applicazioni 
sia lato client che lato server. JavaScript è ormai usato per sistemi distribuiti, 
ed eseguibili su qualsiasi dispositivo che lo supporti, indipendentemente da sistema 
operativo e hardware utilizzati.Abbiamo visto l'utilizzo di database documentali al 
posto dei database relazionali, evidenziandone le principali differenze.Il frutto 
dell'utilizzo di queste tecnologie combinate è una applicazione web eseguibile 
attraverso un semplice browser, ma con caratteristiche e prestazioni quasi o del 
tutto equiparabili ad un'applicazione desktop tradizionale.

La tecnologia sviluppata con questo progetto, data dall'insieme di unicam-product-editor
 e unicam-product-viewer apre le porte ad un nuovo modo di interpretare la visualizzazione
  dei prodotti nell'e-commerce, che fino ad oggi si è basata su tecnologie semplici 
  ed efficaci, ma che nel corso degli anni non hanno mai avuto un vero punto di svolta. 
  La realtà aumentata applicata al configuratore di oggetti 3D è un fattore da non 
  sottovalutare, poiché porta l'esperienza utente ad un livello successivo, trasportando 
  l'utilizzatore in una realtà in cui è possibile osservare un prodotto molto più da 
  vicino di quanto non si possa fare con una semplice immagine. La terza dimensione 
  applicata agli oggetti permette di vedere un prodotto quasi come se fosse davanti 
  a noi, e non attraverso uno schermo. Inoltre, grazie all'approccio con cui è stata 
  sviluppata questa tecnologia, è facile immaginare come il configuratore possa 
  essere usato attraverso qualsiasi dispositivo che supporta le tecnologie elencate 
  in questa tesi, ma che sono comunque distribuite su quasi tutte le piattaforme di 
  maggiore utilizzo.

E' facile immaginare il nostro progetto implementato in una applicazione mobile per arredamenti, 
capace di far visualizzare elementi di mobilio in varie configurazioni, e nella posizione reale 
in cui vorremmo vederli.
Lo stesso vale per una applicazione web di un negozio online di calzature, 
in cui l'utente può scegliere la combinazione di elementi che preferisce, per poi visualizzare 
la scarpa quasi come se fosse davanti a lui, e quasi con la possibilità di indossarla.

Sia unicam-product-editor che unicam-product-viewer sono stati realizzati interamente 
con l'uso di strumenti gratuiti e open-source, il che li rende ancor più facilmente 
utilizzabili e applicabili in diversi contesti e secondo diverse specifiche.L'unico 
punto da tenere bene a mente è quello della manutenzione: le tecnologie, le librerie 
e gli strumenti usati sono di moderno stampo e sono in continua crescita ed evoluzione,
 ed è quindi importante aggiornare entrambi i progetti, in modo tale da mantenerli in
 linea con le novità di tutte le librerie, i framework e gli ambienti che utilizzano JavaScript.