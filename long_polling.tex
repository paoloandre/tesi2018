\chapter{Long Polling}
\label{chap:long_polling}
In questo capitolo si spiega il funzionamento del modello di comunicazione Long Polling, per poi spiegare come è applicato in unicam-product-editor, ed infine compararlo con altre metologie in uso.

L'XMLHTTPRequest Long Polling è, di base, una tecnologia dove il client richiede un'informazione al server senza aspettarsi una risposta immediata.
Questo fa sì che la connessione che esiste fra client e server abbia una connessione di lunga durata, durante la quale il server esegue un'operazione complessa, e solo una volta portata a termine, restituisce una risposta al client inviando i dati richiesti come risposta, o anche solo una notifica.

Innanzitutto il fatto che questa tecnologia sia basata su XMLHTTPRequest ci fa capire che le chiamate che il client effettua nei confronti del server sono delle semplici chiamate Ajax, ma a differenza dell'uso tradizionale di queste chiamate, la gestione degli eventi non viene eseguita dalla parte del client (client-side), ma dalla parte del server (server-side).
Inoltre, a differenza delle chiamate Ajax, che vengono effettuate a intervalli regolari, ad esempio ogni 10 secondi al termine delle quali la connessione al server viene chiusa, in una chiamata di tipo Long Polling, la connessione col server rimane aperta finché il server stesso non invia una risposta, oppure finché non si raggiunge un limite di tempo impostato, detto timeout.

Questo tipo di tecnlogia usato nelle applicazioni non è nuovo, le web chat sono sempre esistite. Quello che è cambiato negli ultimi anni è l’approccio tecnologico a basso livello. Una volta i client facevano richieste HTTP ogni pochi secondi al server richiedendo eventuali messaggi (come ad esempio nel protocollo POP3 per la ricezione delle email) mentre nel Long Polling è il server stesso a notificare il client solamente a fronte di novità.
\newpage
Di seguito uno schema che riassume l'architettura Long Polling, mostrando l'interazione fra un client e un server, e le relative richieste e risposte.
\begin{figure}[h]
	\centering
	\includegraphics[scale=0.7]{Immagini/long_polling.png}
	\caption{Modello di comunicazione Long Polling}
\end{figure}

Questo metodo, assieme ad altri con lo stesso funzionamento di base (ad esempio lo Streaming) insieme costituiscono il modello architetturale Comet, in cui le richieste HTTP di lunga durata che permettono ad un web server di inviare dati ad un browser senza che quest'ultimo li abbia esplicitamente richiesti ne sono la caratteristica principale.

La tecnica del Long Polling, nonostante non sia difficilissima da implementare una volta compresa, non può essere però eseguita in qualsiasi server web proprio per la caratteristica di avere richieste in stato “pending”. I principali server web e application server non offrono questa funzionalità nativa perchè agiscono, differentemente da Node.js, in maniera sincrona. Hanno a disposizione un numero finito di “slot di richieste”, esaurito il quale sono obbligati a rilasciarne qualcuna.
NodeJS grazie al suo modello \emph{event-driven} offre una struttura di basso livello che eccelle in questo tipo di applicazioni. Il meccanismo basato su callback rappresenta il deux ex machina per un architettura basata su Long Polling.

In unicam-product-editor, e più precisamente nella fase di upload di un oggetto 3D, il caso base che si verificherà sarà quello del client che inizia il caricamento della forma attraverso l'interfaccia, inviando una richiesta HTTP al server Node.js, il quale inizierà in sequenza le fasi di upload, conversione in JSON e inserimento del file convertito nel database. Durante questo processo, che per sua natura ha un tempo di esecuzione lungo (essendo i file 3D in formato .obj ed i corrispondenti file JSON una dimensione media nell'ordine dei MegaByte) la richiesta HTTP rimane in una fase di congelamento, finché il server non avrà eseguito tutto il processo, ed invierà al client una risposta (sia essa di successo nell'esecuzione o di errore), o finché non si sarà raggiunto il tempo limite, detto timeout, superato il quale la richiesta si esaurirà.

Il Long Polling HTTP inoltre è utilissimo per creare API affidabili, in quanto le azioni di sincronizzazione e di ascolto possono essere combinate in una stessa richiesta. Nel nostro caso, essendo questo progetto basato su RESTful API, è facile espandere una di queste trasformandola in una Long Polling API, mantenendo comunque la stessa semantica, usando anche questo sistema delle interazioni di tipo \emph{request/response}. Dei timeout brevi possono oltretutto aumentare la solidità delle richieste fra client e server nel caso in cui, ad esempio, l'indirizzo IP di un client cambi come conseguenza del roaming da rete wireless a rete mobile o tethering.

E' per questi motivi che in unicam-product-editor è stata implementata la tecnologia di Long Polling nella fase di upload di una forma 3D.
Il tutto rende il processo di upload non bloccante, ossia permette all'utente, durante l'esecuzione delle operazioni sopra elencate, di continuare ad eseguire altre operazioni sulla applicazione web, mentre il processo viene portato a termine in background.
Per rendere ciò possibile, dobbiamo ricordarci che il progetto è basato su un server Node.js, il quale è di tipo \emph{event-driven}, ed esegue le operazioni in modo asincrono. Proprio per questa caratteristica, la fase di upload di un oggetto 3D non è bloccante nei confronti dell'utente, che può così continuare ad inviare al server normali richieste HTTP.

\section{Implementazione del Long Polling in unicam-product-editor}
In questa fase andremo a vedere l'uso pratico della tecnologia appena descritta nel nostro editor di forme 3D.

Per quanto riguarda le normali chiamate HTTP Ajax, è stata usata jQuery\index{jQuery}, una libreria JavaScript veloce, leggera e ricca di funzionalità.

Prendiamo in esame il Controller Angular su cui si sviluppa la pagina dell'uploader:

\begin{lstlisting}[{caption=HTTP POST /upload}, style=JavaScriptCode]
$http.post('/upload', fd, {
	withCredentials: true,
	headers: {'Content-Type': undefined },
	transformRequest: angular.identity
})
\end{lstlisting}
Per prima cosa si esegue la HTTP POST request relativa all'upload e successiva conversione del file .obj. La chiamata API è indicata subito dopo la funzione \texttt{\$http.post}, ed è \texttt{/upload}.
La risposta che arriverà da parte del server sarà l'endpoint del file JSON convertito, che metteremo nella variabile \texttt{filename}.
Dopo questo si esegue una chiamata HTTP GET, passando come parametro il nome del file appena convertito, per inserire il file JSON nel database.  

\begin{lstlisting}[{caption=HTTP GET /insert}, style=JavaScriptCode]
$http({method: 'GET', url: filename})
	.then(function successGet(filename) {
		$http({method: 'POST',
			url: '/insert',
			data: {shape: filename.data}
		})
	...
\end{lstlisting}

La gestione degli errori avviene tramite la funzione \texttt{.then} della richiesta HTTP: se l'operazione va a buon fine, la risposta del server sarà il codice \texttt{200, OK}, mentre in caso contrario il server risponderà con un codice di errore. In entrambi i casi il risultato è contenuto nel parametro \texttt{response}.

\begin{lstlisting}[{caption=gestione degli errori nella richiesta HTTP}, style=JavaScriptCode]
.then(function successResponse(response){
	console.log(response);
	console.log('file inserted in db');
	$scope.isRouteLoading = false;
	$scope.uploadsuccess = true;
}, function errorResponse(response) {
	console.log(response);
	console.log('error in insert');
});
\end{lstlisting}

\section{Altri metodi di comunicazione fra client e server}
Di seguito vengono elencati, oltre al Long Polling, gli altri metodi di comunicazione fra client e server, con particolare riguardo al modo in cui vengono scambiate le richieste fra i due interlocutori.

\subsection{Normale HTTP}
\begin{enumerate}
	\item Il client richiede una pagina web al server.
	\item Il server calcola la risposta.
	\item Il server invia la risposta al client.
\end{enumerate}
\begin{figure}[h]
	\centering
	\includegraphics[scale=0.4]{Immagini/regular_http.png}
\end{figure}
\subsection{Polling Ajax}
\begin{enumerate}
	\item Il client richiede una pagina web al server usando il normale HTTP.
	\item Il client riceve la pagina web richiesta ed esegue il codice JavaScript contenuto nella pagina, il quale richiede un file al server ad intervalli regolari (es. 0.5 secondi).
	\item Il server calcola ogni risposta e la invia al client, come del normale traffico HTTP.
\end{enumerate}
\begin{figure}[h]
	\centering
	\includegraphics[scale=0.4]{Immagini/ajax_polling.png}
\end{figure}
\subsection{Long Polling Ajax}
\begin{enumerate}
	\item Il client richiede una pagina web al server usando il normale HTTP.
	\item Il client riceve la pagina web richiesta ed esegue il codice JavaScript contenuto nella pagina, il quale richiede un file al server.
	\item Il server non risponde immediatamente con l'informazione richiesta, ma aspetta finché non c'è una nuova informazione disponibile.
	\item Quando l'informazione è disponibile, il server risponde con questa.
	\item Il client riceve la nuova informazione e manda immediatamente un'altra richiesta al server, riavviando il processo.
\end{enumerate}
\begin{figure}[h]
	\centering
	\includegraphics[scale=0.4]{Immagini/ajax_long_polling.png}
\end{figure}
\subsection{HTML5 Server Sent Events (SSE)/EventSource}
\begin{enumerate}
	\item Il client richiede una pagina web al server usando il normale HTTP.
	\item Il client riceve la pagina web richiesta ed esegue il codice JavaScript contenuto nella pagina, il quale apre una connessione al server.
	\item Il server invia un evento al client non appena è disponibile una nuova informazione.
\end{enumerate}
\begin{figure}[h]
	\centering
	\includegraphics[scale=0.4]{Immagini/server_sent_events.png}
\end{figure}
\newpage
\subsection{HTML5 Websockets}
\begin{enumerate}
	\item Il client richiede una pagina web al server usando il normale HTTP.
	\item Il client riceve la pagina web richiesta ed esegue il codice JavaScript contenuto nella pagina, il quale apre una connessione al server.
	\item Il client e il server possono ora scambiarsi messaggi quando sono disponibili nuove informazioni (da qualsiasi lato).
\end{enumerate}
\begin{figure}[h]
	\centering
	\includegraphics[scale=0.4]{Immagini/websockets.png}
\end{figure}