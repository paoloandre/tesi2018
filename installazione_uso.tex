\chapter{Installazione e Uso}
\label{chap:installazione_uso}
In questa sezione andremo a vedere come installare ed avviare unicam-product-editor.
\section{Ambiente locale}
Per prima cosa va scaricato o clonato il codice dalla repository presente su GitHub.
\begin{lstlisting}[style=javaScriptCode]
$ git clone https://github.com/e-xtrategy/unicam-product-editor.git
\end{lstlisting}
Poi, dalla cartella del progetto, installiamo i moduli e le dipendenze necessarie per il funzionamento dell'applicazione
\begin{lstlisting}[style=javaScriptCode]
npm install -g
\end{lstlisting}
Ed infine, per far partire l'applicazione, lanciamo il comando
\begin{lstlisting}[style=javaScriptCode]
npm start
\end{lstlisting}
A questo punto il server è connesso al database in hosting su mLab e in ascolto sulla porta 3000.
Per eseguire l'applicazione è sufficiente aprire il browser e digitare nella barra URL il seguente indirizzo
\begin{lstlisting}[style=javaScriptCode]
http://localhost:3000
\end{lstlisting}
\section{Heroku}
Per aprire l'applicazione rilasciata su Heroku basta aprire il browser e inserire il seguente URL nella apposita barra:
\begin{lstlisting}[style=javaScriptCode]
https://unicam-product-editor.herokuapp.com
\end{lstlisting}